%\documentclass{standalone}
%\usepackage{booktabs}
%\begin{document}
%\begin{tabular*}{1\textwidth}{@{\extracolsep{\fill}} l l}
%  \toprule
%  \midrule
%  1) caudal anterior cingulate ({\tt cACC})  & 17) pars orbitalis ({\tt pORB}) \\
%  2) caudal middle frontal ({\tt cMFG})      & 18) pars triangularis ({\tt pTRI}) \\
%  3) cuneus ({\tt CUN})                      & 19) pericalcarine ({\tt periCAL}) \\
%  4) entorhinal ({\tt ENT})                  & 20) postcentral ({\tt postC}) \\
%  5) fusiform ({\tt FUS})                    & 21) posterior cingulate ({\tt PCC}) \\
%  6) inferior parietal ({\tt IPL})           & 22) precentral ({\tt preC}) \\
%  7) inferior temporal ({\tt ITG})           & 23) precuneus ({\tt PCUN}) \\
%  8) isthmus cingulate ({\tt iCC})           & 24) rosterior anterior cingulate ({\tt rACC}) \\
%  9) lateral occipital ({\tt LOG})           & 25) rostral middle frontal ({\tt rMFG}) \\
%  10) lateral orbitofrontal ({\tt LOF})      & 26) superior frontal ({\tt SFG}) \\
%  11) lingual ({\tt LING})                   & 27) superior parietal ({\tt SPL}) \\
%  12) medial orbitofrontal ({\tt MOF})       & 28) superior temporal ({\tt STG}) \\
%  13) middle temporal ({\tt MTG})            & 29) supramarginal ({\tt SMAR}) \\
%  14) parahippocampal ({\tt PARH})           & 30) transverse temporal ({\tt TT}) \\
%  15) paracentral ({\tt paraC})              & 31) insula ({\tt INS}) \\
%  16) pars opercularis  ({\tt pOPER})        & {}\\
%  \bottomrule
%\end{tabular*}
%\end{document}


\begin{table}[!htb]
\centering
\begin{tabular*}{0.95\textwidth}{@{\extracolsep{\fill}} l l}
 \toprule
 \midrule
 1) caudal anterior cingulate ({\tt cACC})  & 17) pars orbitalis ({\tt pORB}) \\
 2) caudal middle frontal ({\tt cMFG})      & 18) pars triangularis ({\tt pTRI}) \\
 3) cuneus ({\tt CUN})                      & 19) pericalcarine ({\tt periCAL}) \\
 4) entorhinal ({\tt ENT})                  & 20) postcentral ({\tt postC}) \\
 5) fusiform ({\tt FUS})                    & 21) posterior cingulate ({\tt PCC}) \\
 6) inferior parietal ({\tt IPL})           & 22) precentral ({\tt preC}) \\
 7) inferior temporal ({\tt ITG})           & 23) precuneus ({\tt PCUN}) \\
 8) isthmus cingulate ({\tt iCC})           & 24) rosterior anterior cingulate ({\tt rACC}) \\
 9) lateral occipital ({\tt LOG})           & 25) rostral middle frontal ({\tt rMFG}) \\
 10) lateral orbitofrontal ({\tt LOF})      & 26) superior frontal ({\tt SFG}) \\
 11) lingual ({\tt LING})                   & 27) superior parietal ({\tt SPL}) \\
 12) medial orbitofrontal ({\tt MOF})       & 28) superior temporal ({\tt STG}) \\
 13) middle temporal ({\tt MTG})            & 29) supramarginal ({\tt SMAR}) \\
 14) parahippocampal ({\tt PARH})           & 30) transverse temporal ({\tt TT}) \\
 15) paracentral ({\tt paraC})              & 31) insula ({\tt INS}) \\
 16) pars opercularis  ({\tt pOPER})        & {}\\
 \bottomrule
\end{tabular*}
\caption{The 31 cortical labels (per hemisphere) of the Desikan-Killiany-Tourville atlas.
        The ROI abbreviations from the R {\tt brainGraph} package are given in
        parentheses and used
        in later figures.
 }
\label{table:dkt_labels}
\end{table}
